\documentclass[a4paper]{article}
\usepackage[right=2cm,left=4cm,top=2.5cm,bottom=2.5cm,headsep=0.75cm,footskip=1cm]{geometry}
\usepackage[utf8]{inputenc}
\usepackage[spanish]{babel}
\usepackage{amsmath, amsthm, amsfonts, amssymb, mathrsfs}
\usepackage{hyperref,fancyhdr}
\usepackage{booktabs}
\usepackage{multicol}
\setlength{\columnsep}{1cm}
\IfFileExists{lmodern.sty}{\usepackage{lmodern}}{}
\usepackage[T1]{fontenc}
\IfFileExists{lmodern.sty}{\usepackage{lmodern}}{}
\usepackage[T1]{fontenc}
\usepackage[utf8]{inputenc}
\hypersetup{
    pdftitle={Praticas PCTR},
    pdfsubject={PCTR},
    pdfauthor={Carlos Rodrigo Sanbria},
    pdfkeywords={PCTR, JAVA, TEX}
}
\title{Ackermann}
\begin{document}
\pagestyle{fancy}
\fancyhf{}
\renewcommand{\headrulewidth}{0mm}
\cfoot{\thepage}
\setlength{\parskip}{2em}
\newenvironment{Table}
    {\par\bigskip\noindent\minipage{\columnwidth}\centering}
    {\endminipage\par\bigskip}
\newcommand{\question}[1] % This is what you will use to create a new question
{%\refstepcounter{questions} % Increases the questions counter, this can be referenced anywhere with \thequestions
\par\noindent % Creates a new unindented paragraph
\phantomsection % Needed for hyperref compatibility with the \addcontensline command
%\addcontentsline{faq}{questions}{#1} % Adds the question to the list of questions
{\textbf{#1}}%\todo[inline, color=green!40] % Uses the todonotes package to create a fancy box to put the question
\vspace{1em} % White space after the question before the start of the answer
}
    \begin{table}
        \centering
        \begin{tabular}{l*{5}{c}}
            \toprule
            & \multicolumn{5}{c}{\textbf{n}} \\
            \cmidrule(lr){2-6}
            \textbf{m} & \textbf{0} & \textbf{1} & \textbf{2} & \textbf{3} & \textbf{4}\\
            \midrule
            \textbf{0} & 1  & 2     & 3  & 4  & 5   \\
            \textbf{1} & 2  & 3     & 4  & 5  & 6   \\
            \textbf{2} & 3  & 5     & 7  & 9  & 11  \\
            \textbf{3} & 5  & 13    & 29 & 61 & 125 \\
            \textbf{4} & 13 & 65533 & \textbf{$\infty$} & \textbf{$\infty$} & \textbf{$\infty$}\\
            \bottomrule
        \end{tabular}
        \caption{Ackermann}
        \label{table:1}
    \end{table}
\section{Preguntas}
\question{¿Qué observa en relación con esta prueba y esta particularísima función?}\\
Que la función crece muy rapidamente y para valores mayores a (4,1) la función genera números muy grandes.
\question{¿Cree que la concurrencia podría jugar algún papel útil para evaluarla mejor?}\\
No, el resultado de la misma depende de valor anteriores.
\end{document}
