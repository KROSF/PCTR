\documentclass[a4paper]{article}
\usepackage[margin=1in]{geometry}
\usepackage[utf8]{inputenc}
\usepackage[spanish]{babel}
\usepackage{amsmath, amsthm, amsfonts, amssymb, mathrsfs}
\usepackage{hyperref,fancyhdr}
\usepackage{booktabs}% fancy tables
\usepackage{tocloft}% agregar tabla de preguntas 
%\usepackage{multicol}
\usepackage{graphicx}
\usepackage{capt-of}
\setlength{\columnsep}{1cm}
\IfFileExists{lmodern.sty}{\usepackage{lmodern}}{}
\usepackage[T1]{fontenc}
\IfFileExists{lmodern.sty}{\usepackage{lmodern}}{}
\usepackage[T1]{fontenc}
\usepackage[utf8]{inputenc}
\usepackage{tikz}
\usepackage{gnuplottex}
\usetikzlibrary{decorations.pathreplacing}
\author{Carlos Rodrigo Sanabria Flores}
\hypersetup{
    pdftitle={Praticas PCTR},
    pdfsubject={PCTR},
    pdfauthor={Carlos Rodrigo Sanbria},
    pdfkeywords={PCTR, JAVA, TEX}
}
\title{Elipse}
\begin{document}
\pagestyle{fancy}
\fancyhf{}
\renewcommand{\headrulewidth}{0mm}
\cfoot{\thepage}
\setlength{\parskip}{2em}
\newenvironment{Table}
    {\par\bigskip\noindent\minipage{\columnwidth}\centering}
    {\endminipage\par\bigskip}
\newcommand{\question}[1] % This is what you will use to create a new question
{%\refstepcounter{questions} % Increases the questions counter, this can be referenced anywhere with \thequestions
\par\noindent % Creates a new unindented paragraph
\phantomsection % Needed for hyperref compatibility with the \addcontensline command
%\addcontentsline{faq}{questions}{#1} % Adds the question to the list of questions
{\textbf{#1}}%\todo[inline, color=green!40] % Uses the todonotes package to create a fancy box to put the question
\vspace{1em} % White space after the question before the start of the answer
}
\maketitle
\begin{abstract}
    \centering
    Es el lugar geométrico de todos los puntos de un plano, tales que la suma de las distancias a otros dos puntos fijos llamados focos es constante.
\end{abstract}
\section{Formula}
\begin{equation}
    \dfrac{x^2}{a^2} + \dfrac{y^2}{b^2} = 1
\end{equation}
\section{Descripción}
Donde $a > 0$ y $b > 0$ son los semiejes de la elipse, donde si $a$ corresponde al eje de las abscisas y $b$ al eje de las ordenadas la elipse es horizontal, si es al revés, entonces es vertical.
\section{Gráfica}
\begin{tikzpicture}[dot/.style={draw,fill,circle,inner sep=1pt}]
    \def\a{4} % large half axis
    \def\b{2} % small half axis
    \def\angle{-45} % angle at which X is placed
    % Draw the ellipse
    \draw[line width=0.3mm] (0,0) ellipse ({\a} and {\b});
    % Draw the inner lines and labels
    \draw (\a,0) coordinate[] (A);
    \draw (0,\b) coordinate[] (B);
    \coordinate[] (O) at (0,0);
    \draw[line width=0.3mm,red] (O) -- node[below] {a} (A);
    \draw[line width=0.3mm,blue] (O) -- node[left] {b}(B);
    % Nodes at the focal points
    % Node on the rim, connected to foci
    % Brace
    %\draw[decorate,decoration=brace,draw=red] (O) -- (A);
  \end{tikzpicture}
\end{document}