\documentclass[a4paper]{article}
\usepackage[margin=1in]{geometry}
\usepackage[utf8]{inputenc}
\usepackage[spanish]{babel}
\usepackage{amsmath, amsthm, amsfonts, amssymb, mathrsfs}
\usepackage{hyperref,fancyhdr}
\usepackage{booktabs}% fancy tables
\usepackage{tocloft}% agregar tabla de preguntas 
%\usepackage{multicol}
\usepackage{graphicx}
\usepackage{capt-of}
\setlength{\columnsep}{1cm}
\IfFileExists{lmodern.sty}{\usepackage{lmodern}}{}
\usepackage[T1]{fontenc}
\IfFileExists{lmodern.sty}{\usepackage{lmodern}}{}
\usepackage[T1]{fontenc}
\usepackage[utf8]{inputenc}
\usepackage{tikz}
\usepackage{gnuplottex}
\usetikzlibrary{decorations.pathreplacing}
\author{Carlos Rodrigo Sanabria Flores}
\hypersetup{
    pdftitle={Praticas PCTR},
    pdfsubject={PCTR},
    pdfauthor={Carlos Rodrigo Sanbria},
    pdfkeywords={PCTR, JAVA, TEX}
}
\title{Estadisticos}
\begin{document}
\maketitle
\begin{abstract}
    \centering
    Formulas y descripción de los estadísticos usados en la práctica.
\end{abstract}
\pagestyle{fancy}
\fancyhf{}
\renewcommand{\headrulewidth}{0mm}
\cfoot{\thepage}
\setlength{\parskip}{2em}
\newenvironment{Table}
    {\par\bigskip\noindent\minipage{\columnwidth}\centering}
    {\endminipage\par\bigskip}
\newcommand{\question}[1] % This is what you will use to create a new question
{%\refstepcounter{questions} % Increases the questions counter, this can be referenced anywhere with \thequestions
\par\noindent % Creates a new unindented paragraph
\phantomsection % Needed for hyperref compatibility with the \addcontensline command
%\addcontentsline{faq}{questions}{#1} % Adds the question to the list of questions
{\textbf{#1}}%\todo[inline, color=green!40] % Uses the todonotes package to create a fancy box to put the question
\vspace{1em} % White space after the question before the start of the answer
}
\section{Media}
Es una medida de tendencia central. Resulta al efectuar una serie determinada de operaciones con un conjunto de números y que, en determinadas condiciones, puede representar por sí solo a todo el conjunto.
\subsection{Arimética}
Valor característico de una serie de datos cuantitativos
\begin{equation}
    \overline{x} = \dfrac{\sum\limits_{i=1}^n{x_i}}{n}
\end{equation}
\subsection{Geométrica}
De una cantidad arbitraria de números es la raíz n-ésima del producto de todos los números.
\begin{equation}
    \overline{x} = (\prod\limits_{i=1}^n{a_i})^{\frac{1}{n}}
\end{equation}
\subsection{Armónica}
Es igual al recíproco, o inverso, de la media aritmética de los recíprocos de dichos valores.
\begin{equation}
    H = \dfrac{n}{\sum\limits_{i=1}^n{\frac{1}{x_i}}}
\end{equation}
\section{Varianza}
Es una medida de dispersión definida como la esperanza del cuadrado de la desviación de dicha variable respecto a su media.
\begin{equation}
    \sigma^2 = \dfrac{\sum\limits_{i=1}^n{x^2_i}}{n}-\overline{x}^2
\end{equation}
\section{Moda}
Es el valor con mayor frecuencia en una distribución de datos.
Se hablará de una distribución multimodal de los datos adquiridos en una columna cuando encontremos mas de una moda, es decir, dos o mas datos que tengan la misma frecuencia absoluta máxima.
\section{Desviación Típica}
Es una medida de dispersión para variables de razón y de intervalo. Se define como la raíz cuadrada de la varianza de la variable.
\begin{equation}
    \sigma = \sqrt{\sigma^2}
\end{equation}
\end{document}